\chapter{Wstęp do problematyki AI}

Sztuczna inteligencja (AI) to dziedzina informatyki zajmująca się tworzeniem systemów zdolnych do wykonywania zadań, które normalnie wymagałyby ludzkiej inteligencji. Zadania te obejmują m.in. rozpoznawanie obrazów, przetwarzanie języka naturalnego, podejmowanie decyzji w warunkach niepewności oraz tłumaczenie między językami. Jak zauważają Russell i Norvig w swojej fundamentalnej pracy, jest to nauka o agentach, którzy odbierają bodźce z otoczenia za pomocą sensorów i podejmują działania za pomocą efektorów, dążąc do maksymalizacji określonej miary użyteczności \cite{russell2020}.

W ostatnich dekadach obserwujemy gwałtowny rozwój tej dziedziny, co wynika z trzech głównych czynników:
\begin{enumerate}
    \item \textbf{Dostępność danych} -- żyjemy w erze Big Data, gdzie ilość generowanych informacji cyfrowych rośnie wykładniczo, dostarczając "paliwa" dla algorytmów uczących się.
    \item \textbf{Moc obliczeniowa} -- rozwój kart graficznych (GPU) oraz dedykowanych układów tensorowych (TPU) umożliwił trenowanie ogromnych sieci neuronowych w rozsądnym czasie.
    \item \textbf{Nowe algorytmy} -- udoskonalenie metod takich jak wsteczna propagacja błędu czy architektura Transformer.
\end{enumerate}

Współczesne systemy AI opierają się głównie na uczeniu maszynowym (\textit{Machine Learning}) oraz jego poddziedzinie -- uczeniu głębokim (\textit{Deep Learning}). W przeciwieństwie do klasycznego programowania, gdzie człowiek definiuje reguły, w uczeniu maszynowym system samodzielnie ekstrahuje wzorce z dostarczonych danych treningowych. Celem tego raportu jest przedstawienie kluczowych aspektów tej technologii, jej historii oraz wpływu na współczesny świat.