\chapter{Historia i rodzaje AI}

Rozwój sztucznej inteligencji nie był procesem liniowym; dziedzina ta przeżywała okresy wielkiego optymizmu, przeplatane tzw. "zimami AI", kiedy to fundusze na badania były drastycznie obcinane z powodu braku spektakularnych sukcesów. Początki teoretyczne sięgają prac Alana Turinga, który w 1950 roku w czasopiśmie "Mind" zadał kluczowe pytanie "Czy maszyny mogą myśleć?" i zaproponował test (znany dziś jako Test Turinga) mający na celu weryfikację inteligencji maszynowej \cite{turing1950}.

\section{Kamienie milowe rozwoju}
Analizując historię sztucznej inteligencji, można wyróżnić momenty przełomowe, które zmieniały postrzeganie możliwości komputerów. Poniżej przedstawiono wybrane wydarzenia (lista numerowana):

\begin{enumerate}
    \item \textbf{1950} -- Publikacja pracy Alana Turinga, która położyła podwaliny pod filozofię sztucznej inteligencji.
    \item \textbf{1956} -- Konferencja w Dartmouth, zorganizowana przez Johna McCarthy'ego. To właśnie tam oficjalnie ukuto termin "Artificial Intelligence" i wyznaczono kierunki badań na kolejne dekady.
    \item \textbf{1997} -- System Deep Blue firmy IBM wygrywa mecz szachowy z ówczesnym mistrzem świata, Garrim Kasparowem. Był to pierwszy dowód na to, że maszyna może przewyższyć człowieka w grze strategicznej wymagającej planowania.
    \item \textbf{2012} -- Przełom w rozpoznawaniu obrazów dzięki sieci AlexNet, co zapoczątkowało rewolucję Głębokiego Uczenia (\textit{Deep Learning}).
    \item \textbf{2022} -- Publiczne udostępnienie modelu ChatGPT przez OpenAI, co spopularyzowało generatywną sztuczną inteligencję wśród setek milionów użytkowników na całym świecie.
\end{enumerate}

\section{Klasyfikacja systemów AI}
W literaturze przedmiotu oraz w dyskursie publicznym systemy AI dzieli się zazwyczaj ze względu na ich zakres kompetencji i poziom zaawansowania (lista nienumerowana):

\begin{itemize}
    \item \textbf{ANI (Artificial Narrow Intelligence)} -- sztuczna inteligencja wąska. Są to systemy wyspecjalizowane w wykonywaniu jednego, konkretnego zadania, często lepiej niż człowiek. Przykłady to algorytmy grające w szachy, systemy rekomendacji Netflixa czy filtry antyspamowe. Obecnie wszystkie istniejące systemy AI należą do tej kategorii.
    
    \item \textbf{AGI (Artificial General Intelligence)} -- ogólna sztuczna inteligencja. Jest to hipotetyczny system, który posiadałby zdolność uczenia się i rozumienia świata na poziomie zbliżonym do człowieka. AGI potrafiłaby przenieść wiedzę z jednej dziedziny do innej i rozwiązywać problemy, z którymi wcześniej się nie zetknęła.
    
    \item \textbf{ASI (Artificial Super Intelligence)} -- superinteligencja. Termin ten określa intelekt znacznie przewyższający możliwości kognitywne najmądrzejszych ludzi we wszystkich dziedzinach, włączając w to kreatywność naukową, mądrość ogólną i kompetencje społeczne.
\end{itemize}