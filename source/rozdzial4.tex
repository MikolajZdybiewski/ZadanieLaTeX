\chapter{Wnioski}

\section{Podsumowanie merytoryczne}
Sztuczna inteligencja jest dynamicznie rozwijającą się dziedziną, która transformuje niemal każdy aspekt ludzkiego życia. Niniejszy dokument zaprezentował jej krótki rys historyczny, od teoretycznych rozważań Turinga po współczesne modele generatywne. Omówiono również podział na słabą (ANI) i silną (AGI) sztuczną inteligencję oraz wskazano kluczowe przykłady zastosowań praktycznych.

\section{Uzasadnienie doboru narzędzi (LaTeX)}
Do wykonania niniejszego dokumentu świadomie wybrano klasę \texttt{report}. Decyzja ta została podyktowana specyfiką zadania oraz strukturą treści. Klasa \texttt{report} jest optymalna dla prac zaliczeniowych i sprawozdań technicznych z następujących powodów:
\begin{itemize}
    \item \textbf{Struktura hierarchiczna} -- Klasa ta natywnie obsługuje podział na rozdziały (\texttt{chapter}), co pozwala na logiczne wyodrębnienie poszczególnych części tematycznych (Wstęp, Historia, Zastosowania). Jest to funkcja niedostępna w klasie \texttt{article}.
    \item \textbf{Formatowanie} -- W przeciwieństwie do klasy \texttt{book}, klasa \texttt{report} jest domyślnie jednostronna i nie wymusza skomplikowanych reguł typograficznych (np. rozpoczynania rozdziałów zawsze na nieparzystej stronie), co czyni ją idealną do generowania cyfrowych dokumentów PDF.
    \item \textbf{Automatyzacja} -- Wykorzystanie systemu \LaTeX{} pozwoliło na automatyczne wygenerowanie spisu treści, bibliografii oraz numeracji tabel i rysunków, co znacznie podnosi estetykę i czytelność pracy.
\end{itemize}

\section{Repozytorium źródeł}
Zgodnie z wymaganiami zadania, pełny kod źródłowy tego projektu został umieszczony w systemie kontroli wersji. Jest on dostępny w publicznym repozytorium pod adresem:

\begin{center}
% Link do repozytorium github
    \url{https://github.com/MikolajZdybiewski/ZadanieLaTeX}
\end{center}